\section{Einleitung}

F\"ur neuartige Verkehrstechnologien, wie das autonome Fahren, sind zuverl\"assige Kommunikationsmechanismen
zu den Mobilfunknetzen von besonders gro{\ss}er Bedeutung~\cite{IEEE}.
Um die immer weiter steigenden Anforderungen an die Verbindungsqualit\"at einhalten zu k\"onnen, bedarf es
der kontinuierlichen Weiterentwicklung immer effizienterer Kommunikationsverfahren zwischen Endger\"at und Netzwerk.

Bevor solche Verfahren jedoch in der Praxis eingesetzt werden k\"onnen, bedarf es zun\"achst ausgiebiger Tests.
Dies stellt die Wissenschaftler allerdings h\"aufig vor ein Dilemma: W\"ahrend die aussagekr\"aftigsten Ergebnisse nat\"urlich
immer in einer realen Messumgebung erzielt werden k\"onnen, sind derartige Experimente meist sehr aufw\"andig, kostspielig und
auch nur begrenzt reproduzierbar.
Oft kommen daher als Alternative sogenannte Netzwerksimulationen zum Einsatz, welche diese Probleme zwar l\"osen, aber daf\"ur
aufgrund der zahlreichen Vereinfachungen weniger aussagekr\"aftige Ergebnisse liefern k\"onnen.

Um diese Schwachstellen zu beheben, wurde in~\cite{IEEE} ein neues Simulationsverfahren vorgestellt,
welches die Aussagekraft realer Experimente
mit den Vorz\"ugen der Netzwerksimulation vereinen soll. Es handelt sich hierbei um die sogenannte Data-driven Network Simulation (DDNS),
welche das Verhalten von Netzwerken basierend auf real erhobenen Netzdaten simuliert, um so die Aussagekraft von Simulationen zu steigern.

Ein zentraler Baustein innerhalb von DDNS ist dabei die m\"oglichst realit\"atsnahe Prognose von Daten\"ubertragungsraten basierend
auf anderen real gemessenen Netzwerkindikatoren.
Zu diesem Zweck kommen bei DDNS Pr\"adiktionsverfahren des Machine Learning zum Einsatz.
Man erhofft sich durch das Training dieser Verfahren auf realen Netzwerkdaten eine m\"oglichst realistische Prognose der Datenraten und
damit auch m\"oglichst aussagekr\"aftige Simulationsergebnisse.

In diesem Projekt soll nun einmal die Eignung zweier Prognoseverfahren, \textit{Extreme Gradient Boosting} und
\textit{Lineare Regression mit ARMA-Fehlern} anhand real erhobener Netzwerkdaten der Anbieter O2, T-Mobile und Vodafone
im Raum Dortmund untersucht werden.
Zu diesem Zweck werden die Modelle f\"ur jeden der Anbieter getrennt angepasst und anschlie{\ss}end auf ungesehenen Daten
ausgewertet und miteinander verglichen.
