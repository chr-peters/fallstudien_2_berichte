\section{Methodik}

\subsection{Extreme Gradient Boosting}

Extreme Gradient Boosting ist ein Verfahren aus dem Bereich des maschinellen Lernens, welches sich
in den letzten Jahren einer immer gr\"o{\ss}eren Beliebtheit erfreut hat~\cite{XGBoost}.
Die Grundlegende Funktionsweise dieses Verfahrens sei im Folgenden kurz beschrieben.

\subsubsection{Ausgangssituation}

Wir gehen davon aus, dass wir \"uber einen Trainingsdatensatz $\mathcal{D} = \{(\mathbf{x}_i, y_i)\}$
der Gr\"o{\ss}e $\left| \mathcal{D} \right| = n$ verf\"ugen, welcher aus den beobachteten Messungen $\mathbf{x}_i \in \mathbb{R}^m$
und der Zielgr\"o{\ss}e $y_i \in \mathbb{R}$ besteht, deren Wert wir vorhersagen wollen.

Das Ziel des Tree Boosting ist es, den Wert von $y_i$ durch ein Ensemble von Entscheidungsb\"aumen (CART)
vorherzusagen:
\begin{equation}
    \hat{y_i} = \phi(\mathbf{x}_i) =  \sum_{k=1}^K f_k(\mathbf{x}_i), \quad f_k \in \mathcal{F}
\end{equation}
Hierbei ist $\mathcal{F}$ die Klasse der besagten Entscheidungsb\"aume, welche in jedem ihrer $T$ Bl\"atter
einen konstanten Wert vorhersagen: $\mathcal{F} = \{f(\mathbf{x}) = w_{q(x)}\}$, wobei $q: \mathbb{R}^m \rightarrow T$
eine Funktion ist, die der Beobachtung $\mathbf{x}$ eines der $T$ Bl\"atter zuordnet und $w \in \mathbb{R}^T$ der Vektor
der Blattvorhersagen (Gewichte) des Baumes ist.

\subsubsection{Zielfunktion}

Die Zielfunktion, welche w\"ahrend des Trainings zur Anpassung des Modells minimiert wird, setzt sich wie folgt zusammen:
\begin{equation}
    \mathcal{L}(\phi) = \sum_{i=1}^n l(\hat{y}_i, y_i) + \sum_{k=1}^K \Omega(f_k)
\end{equation}
Hierbei ist $l$ eine differenzierbare und konvexe Verlustfunktion, welche Aufschluss \"uber die G\"ute der Vorhersage $\hat{y}_i$
liefert. Ein Beispiel ist der quadratische Fehler, welcher durch $l(\hat{y}_i, y_i) = (\hat{y}_i - y_i)^2$ gegeben ist.
Die Funktion $\Omega$ ist ein sogenannter Regularisierungs- oder Strafterm und ist wie folgt definiert:
\begin{equation}
    \Omega(f) = \gamma T + \frac{1}{2} \lambda \left \lVert w \right \rVert^2
\end{equation}
Das Ziel von $\Omega$ ist es, eine zu hohe Komplexit\"at der einzelnen Entscheidungsb\"aume in der Optimierung zu bestrafen und somit
w\"ahrend des Trainings simplere B\"aume zu bevorzugen. Dies geschieht mit dem Hintergedanken, eine \"Uberanpassung des Modells an
die Trainingsdaten verhindern zu wollen.
Der Parameter $\gamma$ bestraft hierbei die Anzahl der Bl\"atter $T$ eines Entscheidungsbaumes und der Parameter $\lambda$ bestraft
zu gro{\ss}e Gewichte in den einzelnen Bl\"attern.

\subsubsection{Training}

Das Grundprinzip des Boosting ist es, die Ensemble Modelle additiv nach dem Greedy-Prinzip zu trainieren.
Dies funktioniert hier so, dass die einzelnen Entscheidungsb\"aume nicht alle gleichzeitig angepasst werden, sondern
nach und nach zum Ensemble hinzugef\"ugt werden. Jeder Baum, welcher in einem Schritt hinzugef\"ugt wird, wird so trainiert,
dass er die Zielfunktion soweit wie m\"oglich minimiert.

Wenn im Optimierungsschritt $t$ also der Entscheidungsbaum $f_t$ zum Ensemble hinzugef\"ugt wird, ergibt sich die folgende
Verlustfunktion, welche durch $f_t$ minimiert werden soll:
\begin{equation}
    \mathcal{L}^{(t)} = \sum_{i=1}^n l(\hat{y}^{(t-1)}_i + f_t(\mathbf{x}_i), y_i) + \Omega(f_t)
\end{equation}
Die Regularisierungsterme $\sum_{k=1}^{t-1} \Omega(f_k)$ der bereits zum Ensemble hinzugef\"ugten B\"aume wurden hierbei weggelassen,
da sie im Zuge der Optimierung in Schritt $t$ nicht mehr ver\"andert werden k\"onnen.

\subsection{Regression mit ARMA-Fehlern}

\subsection{Validierung}