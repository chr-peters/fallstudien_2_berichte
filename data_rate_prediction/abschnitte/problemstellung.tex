\section{Problemstellung}

\subsection{Datenbeschreibung}
\label{sec:daten}

Die vorliegenden Daten wurden im Zuge mehrerer Testfahrten durch das deutsche LTE-Netz der Netzbetreiber O2,
T-Mobile und Vodafone im Raum Dortmund erhoben~\cite{IEEE}.
Die Testfahrten verliefen \"uber vier zuvor festgelegte Routen, welche sich hinsichtlich der Art ihrer Umgebung unterscheiden:
\begin{itemize}
    \item \textbf{Campus}: Direkte Umgebung der TU Dortmund, Routenl\"ange 3km.
    \item \textbf{Urban}: Stadtbereich, Routenl\"ange: 3km.
    \item \textbf{Suburban}: Vorstadtbereich, Routenl\"ange: 9km.
    \item \textbf{Highway}: Autobahn, Routenl\"ange: 14km.
\end{itemize}
Jede dieser Messfahrten wurde zehnmal wiederholt.
Hierbei wurden sowohl passive Messungen der Netzqualit\"at mithilfe verschiedener Indikatoren, als auch aktive
Messungen der Up- und Downloadraten durchgef\"uhrt.
Die Messungen der Da\-ten\-\"uber\-tra\-gungs\-ra\-ten wurden alle 10s vollzogen, die Messungen der passiven Indikatoren alle 1s.
Um die Da\-ten\-\"uber\-tra\-gungs\-ra\-ten erfassen zu k\"onnen,
wurden Datenpakete zuf\"alliger Gr\"o{\ss}e von \mbox{0.1, 0.5, 1, ..., 10 MB} an einen Server zur Messung \"ubertragen.
Die insgesamt erhobenen Variablen seien in der folgenden Auflistung kurz beschrieben:
\begin{itemize}
    \item \textbf{RSRP:} \textit{Reference Signal Received Power} gibt die Empfangsst\"arke eines Referenzsignals an. Je h\"oher der Wert,
        desto besser ist der Empfang.
    \item \textbf{RSRQ:} \textit{Reference Signal Received Quality} ist ein weiterer Indikator f\"ur die Verbindungsqualit\"at.
        Er wird unter anderem aus dem RSRP berechnet und kann vom Funkmast verwendet werden, 
        um die Notwendigkeit eines Funkmastwechsels absch\"atzen zu k\"onnen.
    \item \textbf{SINR:} \textit{Signal-to-interference-plus-noise Ratio} gibt das Verh\"altnis des tats\"achlichen Signals zum Rauschen
        oder anderen St\"oreinfl\"ussen an.
    \item \textbf{CQI:} \textit{Channel Quality Indicator} ist ein Indikator, welcher Aufschluss \"uber die
        Qualit\"at des \"Ubertragungskanals gibt.
    \item \textbf{TA:} \textit{Timing Advance} gibt den Zeitversatz an, der zur Synchronisation zwischen Up- und Downlink
        verwendet wird. Damit gibt er indirekt Aufschluss \"uber die Entfernung zum Funkmast.
    \item \textbf{f:} Gibt die \textit{Frequenz} des LTE-Signals an.
    \item \textbf{Velocity:} Die Geschwindigkeit, mit der sich das Messger\"at fortbewegt.
    \item \textbf{Cell ID:} Identifiziert eine Zelle im LTE-Netzwerk. Nicht zu verwechseln mit der eNodeB-ID, welche einen Funkmast
        identifiziert. Ein Funkmast kann mehrere Zellen haben.
    \item \textbf{Payload Size:} Die Gr\"o{\ss}e des \"ubertragenen Datenpakets zur Ermittlung der Daten\"ubertragungsrate.
    \item \textbf{Data Rate:} Die gemessene Daten\"ubertragungsrate. Es werden sowohl Upload- als auch Downloadraten gemessen.
\end{itemize}
Zu den passiven Messungen, welche alle 1s durchgef\"uhrt werden, z\"ahlen hierbei RSRP, RSRQ, SINR, CQI, TA, f, Velocity und
Cell ID. Payload Size und Data Rate sind aktive Messungen, welche alle 10s vollzogen wurden.

Die Messungen aller Testfahrten lassen sich insgesamt zu vier verschiedenen Datens\"atzen zusammenfassen, welche auch sp\"ater
zur Bearbeitung der Projektziele verwendet werden:
\begin{itemize}
    \item \textbf{Context}: Dieser Datensatz enth\"alt die sek\"undlich durchgef\"uhrten passiven Messungen der Netzwerkindikatoren.
        Insgesamt enth\"alt dieser Datensatz 68334 Messungen.
    \item \textbf{Cells}: Enth\"alt Messungen des RSRP und RSRQ zu den Nachbarzellen der aktuell verbundenen Zelle.
        Dieser Datensatz enth\"alt insgesamt 93443 Messungen.
    \item \textbf{Upload}: Dieser Datensatz enth\"alt die Messungen der Upload-Raten, welche alle 10s durchgef\"uhrt werden zuz\"uglich
        der Indikatoren aus dem Context-Datensatz zum entsprechenden Zeitpunkt. Insgesamt enth\"alt dieser Datensatz 6180 Messungen.
    \item \textbf{Download}: Analog zum Upload-Datensatz, nur dass hier die gemessenen Down\-load-Ra\-ten erfasst wurden.
        Dieser Datensatz enth\"alt insgesamt 6516 Messungen.
\end{itemize}

\subsection{Zielsetzungen}

\subsubsection{Task I -- Vorhersage der Daten\"ubertragungsraten}

Ein wesentlicher Aspekt von DDNS ist die m\"oglichst realit\"atsnahe Prognose von Daten\"ubertragungsraten basierend auf
den \"ubrigen Netzwerkindikatoren mithilfe von statistischen Vorhersagemodellen.

Das erste Ziel dieses Projektes ist es nun, verschiedene Prognosemodelle
auf diese Problemstellung an\-zu\-wen\-den, und die G\"ute dieser Verfahren zu untersuchen.
Hierbei wird auch analysiert, ob sich das Verhalten der Modelle bez\"uglich der verschiedenen Netzbetreiber und Testfahrtszenarien
unterscheidet.
Weiterhin wird auch die Relevanz der verwendeten Kovariablen f\"ur die Modellvorhersagen untersucht.

\subsubsection{Task II -- Vorhersage der eNodeB-Verbindungsdauern}

Bei den ersten Eins\"atzen von DDNS in~\cite{IEEE} hat sich gezeigt, dass es oft zu gro{\ss}en Vorhersagefehlern kommt, wenn der
Funkmast gewechselt wird (in der Fachsprache hei{\ss}en LTE-Funkmasten auch \textit{eNodeB}).
Eine Idee, um diesem entgegenzuwirken ist, den Zeitpunk des eNodeB-Wechsels vorherzusagen. Kennt man diesen Zeitpunkt, k\"onnte man
diese Information im n\"achsten Schritt dazu verwenden, um die Pr\"adiktionsmodelle zu verbessern.

Das zweite Ziel dieses Projektes ist also, die Restdauer der bestehenden Verbindung zu einer eNodeB und damit indirekt auch den
Wechselzeitpunk zur n\"achsten eNodeB vorherzusagen.
Auch hier wird die G\"ute des Modells anschlie{\ss}end analysiert und das Verhalten bez\"uglich
der verschiedenen Netzbetreiber und Einsatzszenarien, sowie die Relevanz der verwendeten Kovariablen untersucht.
