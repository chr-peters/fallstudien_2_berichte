\section{Problemstellung}

\subsection{Datenbeschreibung}

Die vorliegenden Daten wurden im Zuge mehrerer Testfahrten durch das deutsche LTE-Netz der Netzbetreiber O2,
T-Mobile und Vodafone im Raum Dortmund erhoben~\cite{IEEE}.
Die Testfahrten verliefen \"uber vier zuvor festgelegte Routen, welche sich hinsichtlich der Art ihrer Umgebung unterscheiden:
\begin{itemize}
    \item \textit{campus}: Direkte Umgebung der TU Dortmund, Routenl\"ange 3km.
    \item \textit{urban}: Stadtbereich, Routenl\"ange: 3km.
    \item \textit{suburban}: Vorstadtbereich, Routenl\"ange: 9km.
    \item \textit{highway}: Autobahn, Routenl\"ange: 14km.
\end{itemize}
Jede dieser Messfahrten wurde zehnmal wiederholt.
Hierbei wurden sowohl passive Messungen der Netzqualit\"at mithilfe verschiedener Indikatoren, als auch aktive
Messungen der Up- und Downloadraten durchgef\"uhrt.
Die Messungen der Da\-ten\-\"uber\-tra\-gungs\-ra\-ten wurden alle 10s vollzogen, die Messungen der passiven Indikatoren alle 1s.
Um die Da\-ten\-\"uber\-tra\-gungs\-ra\-ten erfassen zu k\"onnen,
wurden Datenpakete zuf\"alliger Gr\"o{\ss}e von \mbox{0.1, 0.5, 1, ..., 10 MB} an einen Server zur Messung \"ubertragen.
Die insgesamt erhobenen Variablen seien in der folgenden Auflistung kurz beschrieben:
\begin{itemize}
    \item \textbf{RSRP:}
    \item \textbf{RSRQ:}
    \item \textbf{SINR:}
    \item \textbf{CQI:}
    \item \textbf{TA:}
    \item \textbf{f:}
    \item \textbf{Velocity:}
    \item \textbf{Cell ID:}
    \item \textbf{Payload Size:}
    \item \textbf{Data Rate:}
\end{itemize}

\subsection{Zielsetzungen}

\subsubsection{Task I -- Vorhersage der Daten\"ubertragungsraten}

In~\cite{IEEE} wurde ein neuartiger Ansatz der datengetriebenen Simulation von Netzwerken
(Data-driven Network Simulation, \textit{DDNS}) vorgestellt, welcher darauf basiert, dass durch datengetriebene Modelle m\"oglichst
realit\"atsnahe Simulationen von Netzwerken erzeugt werden sollen.
Ein Aspekt dieser Modelle besteht darin, dass Up- und Down\-load\-ra\-ten abh\"angig von den \"ubrigen Netzwerkindikatoren 
m\"oglichst realistisch modelliert werden m\"ussen.
Hierzu werden Pr\"adiktionsmodelle ben\"otigt, welche diese Da\-ten\-\"uber\-tra\-gungs\-ra\-ten entsprechend vorhersagen k\"onnen.
Das erste Ziel dieses Projektes ist es nun, verschiedene Arten von Pr\"a\-dik\-tions\-mo\-dellen im Hinblick auf diese Problemstellung
an\-zu\-wen\-den, und die G\"ute dieser Verfahren zu untersuchen.

\subsubsection{Task II -- Vorhersage der eNodeB-Verbindungsdauern}