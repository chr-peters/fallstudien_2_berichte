\section{Zusammenfassung}

In diesem Projekt wurde erfolgreich gezeigt, dass es prinzipiell m\"oglich ist, Daten\"ubertragungsraten in LTE-Netzen
auf Basis gemessener Netzwerkindikatoren vorherzusagen, was die Idee des DDNS-Verfahrens rechtfertigt.
Das Extreme Gradient Boosting Modell ist basierend auf den Out-of-Sample Auwertungen zu diesem Zweck besser geeignet,
als das etwas weniger komplexe Modell der linearen Regression mit ARMA-Fehlern.

Bei der Vorhersage der eNodeB-Verbindungsdauern konnte zwar ebenfalls gezeigt werden, dass diese auf Basis der Messungen
prinzipiell M\"oglich zu sein scheint, jedoch unterscheiden sich die Ergebnisse f\"ur die jeweiligen Anbieter mitunter stark.
Es bleibt offen, ob und wie die Vorhersage der eNodeB-Verbindungsdauern zu einer Verbesserung der Vorhersage der
Daten\"ubertragungsraten beitragen kann. Ein denkbarer Weg hierzu w\"are die Einbindung der vorhergesagten Verbindungsdauern
als neue Kovariable f\"ur die Datenratenvorhersage. Ob dies dann einen tats\"achlichen Mehrwert liefern w\"urde, bleibt zu \"uberpr\"ufen.
