\section{Zusammenfassung}

In diesem Projekt wurde erfolgreich gezeigt, dass es prinzipiell m\"oglich ist, Daten\"ubertragungsraten in LTE-Netzen
auf Basis gemessener Netzwerkindikatoren vorherzusagen, was die Idee des DDNS-Verfahrens rechtfertigt.
Das Extreme Gradient Boosting Modell ist basierend auf den Out-of-Sample Auswertungen zu diesem Zweck besser geeignet,
als das etwas weniger komplexe Modell der linearen Regression mit ARMA-Fehlern.

Bei der Vorhersage der eNodeB-Verbindungsdauern konnte zwar ebenfalls gezeigt werden, dass diese auf Basis der Messungen
prinzipiell M\"oglich zu sein scheint, jedoch unterscheiden sich die Ergebnisse f\"ur die jeweiligen Anbieter mitunter stark.
Es bleibt offen, ob und wie die Vorhersagen der eNodeB-Verbindungsdauern zu einer Verbesserung der Vorhersage der
Daten\"ubertragungsraten beitragen k\"onnen. Ein denkbarer Weg hierzu w\"are die Einbindung der vorhergesagten Verbindungsdauern
als neue Kovariable f\"ur die Datenratenvorhersage. Ob dies dann einen tats\"achlichen Mehrwert liefern w\"urde, bleibt zu \"uberpr\"ufen.

Bei der \"Ubertragung der vorliegenden Projektergebnisse auf zuk\"unftige Messungen ist allerdings trotz aller getroffenen Vorkehrungen
bei der Modellvalidierung noch Vorsicht geboten. Man darf nicht vergessen, dass im Zuge der Datenerhebung lediglich vier verschiedene
fest definierte Routen abgefahren wurden. Es wurde zwar penibel darauf geachtet, dass die Modelle bei der Auswertung keine
Informationen aus zuk\"unftigen Messungen verwenden konnten, jedoch wurden in den Testfahrten 8-10 die gleichen Routen erneut
gefahren, die auch schon in den Trainingsfahrten 1-7 gefahren wurden.
\"Uber das Verhalten der Modelle auf neuen, noch nicht befahrenen Routen, kann auf Basis der vorliegenden Auswertung also keine
belastbare Aussage getroffen werden. Dies k\"onnte aber ein Teil von zuk\"unftigen Analysen werden,
welche m\"oglicherweise anderen Validierungsverfahren einsetzen,
die diese Begebenheiten ber\"ucksichtigen k\"onnen.
