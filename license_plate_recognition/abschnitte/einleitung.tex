\section{Einleitung}

Die automatisierte Erkennung von Nummernschildern aus Bilddaten
ist ein wichtiger Bestandteil vieler moderner Verkehrssysteme
und kommt beispielsweise in Parkh\"ausern, Mautstellen
oder bei der Identifikation gestohlener Fahrzeuge zum
Einsatz~\cite{silva2018a}.

Das Ziel dieses Projektes ist es, einen Prototypen f\"ur ein solches
Erkennungssystem zu entwickeln, welcher in der
Lage sein soll, erfolgreich Nummernschilder aus Bilddaten erkennen
und auszulesen zu k\"onnen.

Zu diesem Zweck wird eine zweistufige Vorhersagepipeline konstruiert,
die ausgehend von einer Bilddatei im ersten Schritt die Position des
Nummernschildes bestimmt und im zweiten Schritt anhand der zuvor ermittelten
Position die Zeichen des Nummernschildes ausliest.

Bei der Positionsbestimmung des Nummernschildes kommen sogenannte
Convolutional Neural Networks zum Einsatz, eine spezielle Art von
Neuronalen Netzen, deren Grundlagen in Abschnitt~\ref{sec:grundlagen}
beschrieben werden. Zum Auslesen der Zeichen werden die
Programmbibliotheken OpenCV~\cite{opencv_library} und Tesseract~\cite{smith_2007}
verwendet.

Ein \"Uberblick \"uber das vorliegende Datenmaterial, welches
bei der Erstellung der Pipeline verwendet wurde, wird in
Abschnitt~\ref{sec:Datenbeschreibung} gegeben.
Der gesamte Aufbau der Pipeline wird in Abschnitt~\ref{sec:pipeline}
erl\"autert, im Anschluss werden die Ergebnisse in Abschnitt~\ref{sec:ergebnisse}
beschrieben und die St\"arken sowie die Schw\"achen des entwickelten
Prototypen diskutiert.

Der gesamte Quellcode dieses Projekts ist Open Source und kann
unter \url{https://github.com/cxan96/license_plate_detection}
abgerufen werden.
