\section{Ergebnisse}
\label{sec:ergebnisse}

Als Basis f\"ur eine Ergebnisdiskussion sind in
Abbildung~\ref{fig:modellvorhersagen} die vorhergesagten Nummernschilder
f\"ur 10 Testbilder zu sehen, die alle weder beim Training noch bei der
Validierung zum early stopping verwendet wurden.

\begin{figure}
    \centering
    \begin{subfigure}{0.31\textwidth}
        \includegraphics[width=\textwidth]{abbildungen/prediction_01.pdf}
        \subcaption{Vorhersage: BB135B0}
    \end{subfigure}
    \begin{subfigure}{0.31\textwidth}
        \includegraphics[width=\textwidth]{abbildungen/prediction_02.pdf}
        \subcaption{Vorhersage: S81QAK}
    \end{subfigure}
    \begin{subfigure}{0.31\textwidth}
        \includegraphics[width=\textwidth]{abbildungen/prediction_03.pdf}
        \subcaption{Vorhersage: RK384AL}
    \end{subfigure}
    \begin{subfigure}{0.31\textwidth}
        \includegraphics[width=\textwidth]{abbildungen/prediction_04.pdf}
        \subcaption{Vorhersage: SC7}
    \end{subfigure}
    \begin{subfigure}{0.31\textwidth}
        \includegraphics[width=\textwidth]{abbildungen/prediction_05.pdf}
        \subcaption{Vorhersage: R3RAAUT}
    \end{subfigure}
    \begin{subfigure}{0.31\textwidth}
        \includegraphics[width=\textwidth]{abbildungen/prediction_06.pdf}
        \subcaption{Vorhersage: -}
    \end{subfigure}
    \begin{subfigure}{0.31\textwidth}
        \includegraphics[width=\textwidth]{abbildungen/prediction_07.pdf}
        \subcaption{Vorhersage: KUKK}
    \end{subfigure}
    \begin{subfigure}{0.31\textwidth}
        \includegraphics[width=\textwidth]{abbildungen/prediction_08.pdf}
        \subcaption{Vorhersage: SS7777XJX}
    \end{subfigure}

    \begin{subfigure}{0.31\textwidth}
        \includegraphics[width=\textwidth]{abbildungen/prediction_09.pdf}
        \subcaption{Vorhersage: CSG91}
    \end{subfigure}
    \begin{subfigure}{0.31\textwidth}
        \includegraphics[width=\textwidth]{abbildungen/prediction_10.pdf}
        \subcaption{Vorhersage: BBBSS223355}
    \end{subfigure}
    \caption[Modellvorhersagen]{Modellvorhersagen f\"ur 10 ungesehene Testbilder.
        Die vorhergesagten Nummernschilder sind durch rote Rechtecke markiert.}
    \label{fig:modellvorhersagen}
\end{figure}

Was den ersten Teil der Pipeline angeht, also die Bildsegmentierung
mithilfe eines CNN, so l\"asst sich erkennen, dass das CNN f\"ur jedes
der Bilder den relevanten Bereich, der das Nummernschild enth\"alt,
korrekt vorhersagen konnte.
Selbst die sehr abgedunkelten Lichtverh\"altnisse in Bild (i) konnten
die Vorhersage nicht negativ beeintr\"achtigen.
Dies kann als Anzeichen gedeutet werden, dass es sich hier um eine
sehr vielversprechende Technologie handeln k\"onnte, was das Auffinden
der Nummernschilder angeht.
Die Vorhersagen k\"onnen also als proof-of-concept f\"ur den ersten
Teil der Pipeline dienen.

Es muss allerdings beachtet werden, dass sich alle Nummernschilder in
Abbildung~\ref{fig:modellvorhersagen} hinreichend gut durch
achsenparallele Rechtecke modellieren lassen, was ja wie
in~\ref{sec:bildsegmentierung} beschrieben, quasi eine Grundannahme f\"ur
die Berechnung der umschlie{\ss}enden Rechtecke war.
F\"ur die Funktionsweise der Bildsegmentierung auf Daten, f\"ur die dies
nicht zutrifft, kann also anhand der hier vorgenommenen Betrachtung
keine Aussage getroffen werden.

Was den zweiten Teil der Pipeline, also die Texterkennung angeht, so
fallen leider ein paar kleinere M\"angel ins Auge.
Lediglich bei Bild (a) wurde das Nummernschild
korrekt ausgelesen. Ansonsten sind die Vorhersagen von einigen kleinen
Schwachstellen gepr\"agt.
Ein Fehler, der h\"aufig vorkommt, ist das scheinbar
willk\"urliche Verdoppeln gelesener Zeichen, wie es beispielsweise
in Bild (h) und auch in Bild (j) zu sehen ist.
Dar\"uber hinaus passiert es wie in Bild (d) und (g) h\"aufig, dass
einige Zeichen garnicht erst erkannt werden.
In Bild (f) wurde sogar, obwohl der Bereich des Nummernschildes zuvor
richtig ausgelesen wurde, kein einziges Zeichen erkannt.

Es l\"asst sich insgesamt also erkennen, dass die Texterkennung
der aktuelle Problempunkt in der Pipeline ist.
Im Folgenden sind daher einige Ideen aufgef\"uhrt, wie dieser Teil
der Pipeline in Zukunft noch verbessert werden kann.

\subsection{Ideen zur Verbesserung der Texterkennung}
\label{sec:verbesserungen}

\paragraph{Framework zur iterativen Verbesserung}

Bevor konkrete Verbesserungsans\"atze eingebracht werden k\"onnen, bedarf
es zun\"achst eines Frameworks, also einer allgemeinen Vorgehensweise,
wie zielgerichtet auf eine Verbesserung der Pipeline hingearbeitet werden kann.
Ein m\"oglicher Ansatz ist es, sich zuerst auf eine Metrik festzulegen,
die die Qualit\"at
der Pipeline beschreibt und dann anhand dieser Metrik m\"ogliche
\"Anderungen und Verbesserungen zu beurteilen.
Neuerungen w\"urden nur dann akzeptiert werden, wenn sie auch zu einer
Verbesserung der zu Anfang festgelegten Metrik f\"uhren.

Eine m\"ogliche Metrik, die in diesem Kontext sinnvoll erscheint,
ist die Levenshtein-Distanz, die erstmals in~\cite{levenshtein}
eingef\"uhrt wurde.
Die Levenshtein-Distanz gibt an, wieviele Einf\"uge-, L\"osch- und
Ersetz-Operationen notwendig sind, um eine Zeichenkette in eine
andere zu \"uberf\"uhren.
In der hier vorliegenden Situation k\"onnte man also die
Levenshtein-Distanzen zwischen den Modellvorhersagen und den
tats\"achlichen Nummernschildern messen und so eine Aussage \"uber
die Qualit\"at der Pipeline erhalten.

Zur Verbesserung der Texterkennung k\"onnte man nun also einen Teil der
Daten beiseite legen, beispielsweise die Bilder in Abbildung~\ref{fig:modellvorhersagen},
und sich als Ziel setzen, die mittlere Levenshtein-Distanz der
Pipeline auf diesen Bildern zu minimieren.
Hierbei k\"onnte man dann iterativ vorgehen: Hat man eine Idee zur
Verbesserung der Pipeline entwickelt, so wird diese anhand der
mittleren Levenshtein-Distanz auf den Testdaten evaluiert.
Nur dann, wenn die mittlere Levenshtein-Distanz durch das Einbringen
dieser Idee verkleinert werden konnte, wird die Idee in die Pipeline
aufgenommen.

Kurz angemerkt sei an dieser Stelle noch, dass man sich bei dieser
Vorgehensweise m\"oglicherweise anf\"allig f\"ur Overfitting
machen kann, da man seine Entscheidungen ma{\ss}geblich von einigen zuvor
selektierten Bildern abh\"angig macht, auf denen die mittlere
Levenshtein-Distanz berechnet wird.
Dies k\"onnte aber dadurch kompensiert werden, dass man eine sehr
vielseitige Bandbreite von Bildern zu dieser Validierung heranzieht,
die m\"oglichst viele der denkbaren Szenarien abdecken.

\paragraph{Einfache Testf\"alle als \glqq Sanity Checks\grqq{}}

Anhand der Vorhersagen auf den Beispielbildern f\"allt auf, dass es
m\"oglicherweise noch unentdeckte Fehler bei der Implementierung der
Texterkennung geben k\"onnte. Insbesondere das h\"aufige doppelte Erkennen
von Zeichen einerseits, sowie das h\"aufige Auslassen von Zeichen
andererseits,
deuten darauf hin. Um in Zukunft m\"ogliche \glqq Bugs\grqq{} ausschlie{\ss}en
zu k\"onnen, ist es sinnvoll, ein paar einfache Tests einzuf\"uhren,
die die Pipeline erf\"ullen muss.

Beispielsweise w\"are denkbar, die Texterkennung mit einzelnen Zeichen zu
testen, um sicherzustellen, dass diese korrekt erkannt werden.
Darauf aufbauend k\"onnte man auch ein paar sehr klar erkennbare
Nummernschilder als Testf\"alle einbringen, die auf jeden Fall von
der Pipeline korrekt erkannt werden m\"ussen.
M\"ochte man dann eine Verbesserung in die Pipeline einbringen, so
muss diese Verbesserung zun\"achst alle Testf\"alle korrekt erf\"ullen.
Dieses Konzept bezeichnet man in der Software-Entwicklung auch als
Regressions-Tests~\cite{testing}.

\paragraph{Wechsel der Technologie}

Ein weitere Punkt, der nicht vergessen werden darf, ist die Wahl der
Technologie f\"ur die Texterkennung.
Es w\"are beispielsweise denkbar, anstelle der Kombination von
Vorverarbeitungsschritten mit OpenCV und der anschlie{\ss}enden
Erkennung durch Tesseract, ebenfalls CNNs zur Erkennung der
Zeichen einzusetzen.
Dieses Vorgehen kam beispielsweise in~\cite{silva2018a} zum Einsatz und
hat zu sehr guten Ergebnissen gef\"uhrt.

Dennoch ist anhand der Vorhersagen auf den Beispielbildern
wohl davon auszugehen, dass das volle Potenzial des bisherigen
Ansatzes noch nicht vollends ausgesch\"opft werden konnte.
Ein kompletter Technologiewechsel, und damit eine komplette
Neuimplementierung der Texterkennung, sollte daher eher als eine
Art letzter Ausweg gesehen werden.
